% Chapter 7

\chapter{CONCLUSIONS} % Write in your own chapter title
\section{Contributions}
The proposed model generates highlights based on the scorecard and excitement features. The replays and advertisements are considered to be undesirable hence they are detected and removed. Key events like 4s, 6s, wicket are identified and categorized as highlights. Highlights based on the crowd cheering and excitement of the commentators are also generated. In order to have well defined boundaries, Events like bowling, commentators, interviews, field view are detected and shot boundaries are detected and segmented.

\section{Future Work}
The tasks which needs further exploration are as follows:
\begin{itemize}
    \item The proposed system is designed for 20 over match. The system can also be generalized for 50 over match. This can be challenging because of its longer duration.
    \item Separate highlight video for each key event can be done. \item Different broadcasters will have scorecard in different formats.Instead of specifying the scorecard's position, detection of the scorecard can be automated.
    \item Parallel processing can be done to improve the execution time.
\end{itemize}


